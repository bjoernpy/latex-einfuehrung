\documentclass[11pt, a4paper]{article}
\usepackage[utf8]{inputenc}
\usepackage[T1]{fontenc}
\usepackage{lmodern}
\usepackage[ngerman]{babel}

% leeres Datum setzten, da der default Wert das heutige Datum ist
\title{Übung 1}
\author{}
\date{}

\begin{document}

\maketitle
\tableofcontents

\section{Verschiedene Überschriften}
\subsection{Können hier verwendet werden}
\subsubsection{Und es kann so aussehen}

In einem Absatz können \\ Zeilenumbrüche gemacht werden

\noindent Der Text kann \textit{kursiv}, \textbf{Fett}, \underline{unterstrichen} oder \textbf{\textit{\underline{auch alles zusammen sein}}}
 
\subsection{Zählen geht von selbst}

\begin{enumerate}
\item Mit
\item aufsteigender
\item Zählweise
\end{enumerate}

\begin{itemize}
\item Oder
\item auch
\item ohne
\item Zählung
\end{itemize}
 
\end{document}
